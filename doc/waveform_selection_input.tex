\section{Waveform selection criteria}

Here we are trying to establish some standards for excluding waveform fits produced by CAP. The motivation is toward publication-quality figures.

\begin{enumerate}

\item Before doing any manual selection, make sure that
\begin{enumerate}
\item you have the right source duration (\refFig{fig:cap_dur}); you may need to manually specify -L, especially for small events

If your magnitude is above 4, then you do not need to do this.

\item you are using L1 norm (this is the default)

\item the relative plotted amplitudes of body and surface waves are scaled correctly. This is just a plotting issue to make sure you can see the shapes of the waveforms. (But make sure you are looking at absolute amplitudes!)

\item {\bf you have set the time shifts to the ``correct'' ranges}. This is specified in the -S flag and also influenced by entries in the weight.dat file. {\bf In the case of P waves, you may have to manually specify the onset time.}

In the case of surface waves, you can specify a systematic time shift in the weights file if, for example, all time shifts appear to be systematically shifted from zero. This is equivalent to acknowledging that the structural model is uniformly slow (or fast) in all directions.

Allowed time shifts depends on (examples given here are based on inversions in Alaska):
%
\begin{enumerate}
\item Filter applied: For body waves (higher frequency content) you need to use a shorter time-shift compared to the surface wave. Example, 1--10 seconds body waves generally need $\sim$2 seconds of maximum shift; 16--50 seconds for surface waves can require upto 10 seconds of time-shift.
\item Distance range: Absolute value of time-shifts generally increases with epicentral distance (longer paths mean longer accumulation of time shift between 1D model and actual earth model). For example, time-shifts for surface wave (16--40 sec) for stations upto 100 km would be less than 5 seconds, but for stations at 400--500 km shifts could be around 10--12 seconds.
\item Source duration: In special cases, like VLFE (very low frequency earthquakes) which has much longer source duration (10-15 seconds for \magw{3.8} VLFE in interior Alaska as compared to 1 second duration for normal tectonic event of comparable magnitude) the time-shifts for surface wave (20--50 seconds) can go up to 18~seconds.
\end{enumerate}

\label{item:tshift}

\item your current moment tensor appears pretty close to correct

This requires having sufficient signal-to-noise levels within the chosen bandpass.

{\bf Note:} If there is no reliable solution, then use the best-quality P polarities.
%These tend to be stations that are closest to the event. (Nodal stations are avoided.)

\end{enumerate}

\item A published set of waveforms in a paper may be a subset of waveform fits. But the inversion should be done on a much larger set of waveforms that would be presented either in catalog results (e.g., ScholarWorks) or in a supplement. It's okay to have ugly fits in the ``final'' version for the catalog.
%Figures in papers need to be ``perfect'' -- they can show the solution derived from a full ``messy'' set of waveforms, but the paper version can show a subset of fits, all of which are good. (Note: do not run the inversion with the subset -- just show the fits.)

%==========================================

\item Here are our standards for catalog figures. {\bf Keep in mind that these will be most effective when you are in the ballpark of the solution. This fine-tuning should only be done for the ``final'' solution.} Also keep in mind that we may choose to plot the green waveforms (\verb+keepBad+ in \verb+cap_plt.pl+) or not. (It is useful for us to see the waveforms that we turn off, but it may be distracting to others.) These criteria are driven by amplitude anomalies, since our misfit function is amplitude-based and will be most affected by spurious amplitudes (even when using L1).

\medskip\noindent
{\bf PHASE 1 (getting in the ballpark of the solution)}:
%
\begin{enumerate}
\item Fix the magnitude and depth to the earthquake-catalog-listed magnitude and depth.

\item Use a ``sensible'' distance range for selecting stations. For example, for small magnitude event ($\mw < 3$) one should set the maximum distance for selecting stations to 200~km. For intermediate magnitude ($\mw > 3.5$) events one can go up to 500~km. This range might also vary depending on the crustal structure and topography. Also keep track of the available green's function (most are premade up to 500~km - \verb+/store/wf/FK_synthetics/+)

\item Start by excluding the stations with the largest amplitude differences. Such differences may prevent you from viewing the waveform fits, since they are scaled by absolute amplitudes.

\item Exclude the set of stations at the largest epicentral distance above which stations all have poor SNR.
\end{enumerate}

\medskip\noindent
{\bf PHASE 2}:
\begin{enumerate}
\item Perform a magnitude search and depth search.

Changing either one of these can have a major impact on the solution!

\item Consider adjusting time shifts, including P arrival times, and P polarities.

\item Exclude windows based on amplitude differences.
%
\begin{enumerate}
\item Set weight zero for all {\bf surface wave} waveforms whose amplitude differences are \makebox{$| \rmd\ln A | \ge A_s$}; these could be affecting the misfit function in a negative manner.
\begin{itemize}
\item $A_s = 1.5$ for regional Alaska inversions
\end{itemize}

\item Set weight zero for all {\bf body wave} waveforms whose amplitude differences are \makebox{$| \rmd\ln A | \ge A_b$}; these could be affecting the misfit function in a negative manner.
\begin{itemize}
\item $A_b = 2.5$ for regional Alaska inversions (or for 2012 Nenana triggered event, which is full waveforms at the ``body wave'' periods)

\end{itemize}

Because surface waves are filtered at longer periods, which should be less sensitive to inaccuracies of the 1D velocity model, we have a stronger rejection criterion for fitting amplitudes (1.5 vs 2.5).

Do this incrementally: remove the largest amplitude anomalies, then rerun, etc.

\end{enumerate}

\item Set weight zero for all waveforms that max out the time shift. Or try to adjust the allowable time shift; see Point~\ref{item:tshift} above.

%\item If Surf V is set to weight zero, then set Surf R weight to zero. The idea here is that if you are not capturing the Rayleigh wave motion on one of the components, you should be cautious about what is being fit on the other component. [But note: We can get strong VR ratios that are caused by shallow structure.] 

\item If PV is set to weight zero, then also set PR weight to zero.

(Note: This means you can have PV only, but not PR only.)

%\item If a Love window is set to weight zero, then set the other two Rayleigh wave windows to zero if they ``look bad''.

%The motivation is that the problem that affected the prior-removed window also affects the remaining windows.

%\item If the Rayleigh wave windows are set to weight zero, then set the Love wave window to zero if it ``looks bad''.

%The motivation is that the problem that affected the prior-removed window also affects the remaining windows.

\item If, among the 5 windows, only the Surf T window is matching, then reject that station. If no other component is matching, then again it raises question on the quality of data. In general, from my (Vipul) experience, SH gives a good match for a variety of solution. Most of the time it is the Body and Rayleigh that nails down the solution. However, one can keep such station only if it is much needed (Number of stations $< 10$).

%\item If a station has no fits (that is, all weights set to zero) and is ``complete garbage,'' then cut it completely. We never need to look at it again. (Manually cut the row from the weight file.) This is probably an unusual occurrence.

%\item If a station has no fits (that is, all weights set to zero) but the waveforms look ``in the ballpark,'' then perhaps we leave these in the plots. (So do not cut them from the weight file.)

\item Ideally we would want to use both the body waves and surface waves, however, there might be cases when either most of the body waves or surface waves need to be thrown out. In case you can not use body waves at $\sim$90\% of the stations, then perform a surface waves only inversion. Same goes for the surface waves.

Motivation: It looks odd to have a large set of body waves with a single surface wave (and vice versa).

%\item If the data and syn waveforms fit in a window, then it should be “on”. In other words, we should not see any green waveforms fitting red waveforms, unless there is some other reason to be suspicious. (Or unless it is Rayleigh wave window where the other component does not fit.)

\item Time shifts for Love and Rayleigh waves should be ``similar,'' since the propagation paths are sampling similar Earth structure (Love shallower than Rayleigh). 

Use this to decide whether to turn off certain waveforms and whether to refine allowable time shifts. [NOTE THIS IS NOT A ``RULE''.] For a publication-quality solution, the time shifts should vary systematically as a function of azimuth. This can be detected from examination of the spider plot time shifts.
\end{enumerate}

\medskip\noindent
{\bf PHASE 3}:

\begin{enumerate}
\item Perform final magnitude and depth search.

\item Use automatic CC thresholding.
\begin{enumerate}
\item If a cross-correlation value is less than 10 in a Rayleigh wave window, then cut all surface waves.
\item If a cross-correlation value is less than 10 in a Love wave window, then cut it.
\end{enumerate}

\end{enumerate}


\end{enumerate}

% THING TO THINK ABOUT
%
%Ideally there should be some quality control prior to rotating components to RT. For example, a zeroed E component will lead to erroneous R and T components, but it will be very hard to identify that both R and T are erroneous.
%
% --> pyseis fix to zero out BOTH R and T component waveforms if E or N is zero
